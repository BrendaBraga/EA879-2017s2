\documentclass{beamer}
%
% Choose how your presentation looks.
%
% For more themes, color themes and font themes, see:
% http://deic.uab.es/~iblanes/beamer_gallery/index_by_theme.html
%
\mode<presentation>
{
  \usetheme{Madrid}      % or try Darmstadt, Madrid, Warsaw, ...
  \usecolortheme{default} % or try albatross, beaver, crane, ...
  \usefonttheme{default}  % or try serif, structurebold, ...
  \setbeamertemplate{navigation symbols}{}
  \setbeamertemplate{caption}[numbered]
}

\usepackage[english]{babel}
\usepackage[utf8x]{inputenc}
\usepackage{graphicx}
\graphicspath{{./fig/}}

\title[02-Arquiteturas]{EA879 -- Introdução ao Software Básico\\Arquiteturas}
\author{Tiago F. Tavares}
\institute{FEEC -- UNICAMP}
\date{Aula 02 -- 10/agosto/2017}

\begin{document}

\begin{frame}
  \titlepage
\end{frame}

% Uncomment these lines for an automatically generated outline.
%\begin{frame}{Outline}
%  \tableofcontents
%\end{frame}

\section{Introdução}

\begin{frame}{Objetivos}
  \Large
  \begin{itemize}
    \item Entender como elementos lógicos se relacionam para gerar processadores
      de propósito geral.
    \item Entender como um simulador funciona.
    \item Analisar código C, assembly e código de máquina e definir como eles
      podem ser equivalentes ou diferentes.
  \end{itemize}
\end{frame}


\end{document}
