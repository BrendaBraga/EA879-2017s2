\documentclass{beamer}
%
% Choose how your presentation looks.
%
% For more themes, color themes and font themes, see:
% http://deic.uab.es/~iblanes/beamer_gallery/index_by_theme.html
%
\mode<presentation>
{
  \usetheme{Madrid}      % or try Darmstadt, Madrid, Warsaw, ...
  \usecolortheme{default} % or try albatross, beaver, crane, ...
  \usefonttheme{default}  % or try serif, structurebold, ...
  \setbeamertemplate{navigation symbols}{}
  \setbeamertemplate{caption}[numbered]
}

\usepackage[english]{babel}
\usepackage[utf8x]{inputenc}
\usepackage{graphicx}
\usepackage{array}

\title[11-contexto-e-preempcao]{EA879 -- Introdução ao Software
Básico\\Revezamento de Processos}
\author{Tiago F. Tavares}
\institute{FEEC -- UNICAMP}
\date{Aula 11 -- 15/agosto/2017}

\begin{document}

\begin{frame}
  \titlepage
\end{frame}

% Uncomment these lines for an automatically generated outline.
%\begin{frame}{Outline}
%  \tableofcontents
%\end{frame}

\section{Introdução}

\begin{frame}{Objetivos}
  \Large
  \begin{itemize}
    \item Entender por que programas diferentes não conseguem acessar a memória
      RAM dedicada aos outros.
    \item Entender como é possível alternar entre dois programas sem que um
      influencie o outro.
  \end{itemize}
\end{frame}


\begin{frame}[fragile]{Simuladores}
  \centering
  \Large
  Junto ao seu grupo, faça uma lista com todos os elementos (exemplo: program
  counter) que devem ser
  incorporados a um simulador para que ele consiga executar programas compilados
  para alguma arquitetura.
\end{frame}


\begin{frame}[fragile]{Memória}
  \centering
  \large
  Associe os tipos aos conceitos:
  \begin{tabular}{|c|p{5cm}|} \\ \hline
  int & Posição de memória que contém o endereço de uma posição de memória onde
    está armazenado um número. \\ \hline
  int* & Posição de memória onde está armazenado um número. \\ \hline
  void* & Posição de memória que contém o endereço de uma posição de memória
    cujo conteúdo é desconhecido. \\ \hline
  \end{tabular}
\end{frame}


\begin{frame}[fragile]{Memória}
  \centering
  \large
  Gostaríamos de isolar a memória que é usada por dois programas diferentes, de
  forma que um não possa acessar a memória do outro. Discuta com seu grupo a
  seguinte solução: o sistema de administração de memória atribui um
  \textit{offset} a todos os pedidos de conteúdo de memória de um programa. Esse
  \textit{offset} é único para cada programa.
\end{frame}

\begin{frame}[fragile]{Memória}
  \centering
  \large
  Gostaríamos de isolar a memória que é usada por dois programas diferentes, de
  forma que um não possa acessar a memória do outro. Discuta com seu grupo a
  seguinte solução: o sistema de administração de memória atribui um
  \textit{offset} a todos os pedidos de conteúdo de memória de um programa. Esse
  \textit{offset} é único para cada programa. Além disso, o sistema de
  administração sabe quanta memória cada programa usa, e restringe
  intencionalmente o acesso a memória fora do segmento pré-determinado.
\end{frame}


\begin{frame}[fragile]{Memória}
  \centering
  \large
  \textit{... o sistema de
  administração sabe quanta memória cada programa usa, e restringe
  intencionalmente o acesso a memória fora do segmento pré-determinado.}

  Junto ao seu grupo, atribua um nome à falha referente à tentativa de acesso
  ilegal a um segmento de memória não-autorizado.
\end{frame}


\begin{frame}[fragile]{Segmentação}
  \centering
  \large
  Faça um desenho mostrando a memória física de um computador. Um programa está
  operando no segmento que vai dos endereços físicos 0 a 49, e outro está operando no
  segmento que vai dos endereços físicos 50 a 99. Ambos executam o código:
  \begin{verbatim}
  int *a = 0;
  (*a) = 12;
  \end{verbatim}

  Mostre o conteúdo da memória física do computador (assumindo que ela foi
  inicializada em zero) ao fim dos dois programas.
\end{frame}


\begin{frame}[fragile]{Processos}
  \centering
  \large
  Dois programas diferentes, em um computador com sistema operacional moderno,
  operam em espaços de memória diferentes e não precisam ``saber'' um da
  existência do outro. Cada programa executando num computador é chamado de
  \textit{processo} -- no sentido de ser algo que pode ser executado por um
  processador.
\end{frame}

\begin{frame}{Pausando um simulador}
  \large
  Suponha que um simulador de microprocessador possa ser
  ``pausado'' antes de ser desligado, e
  depois retome quando ligado novamente. Quais elementos (de
  hardware e software) devem ser restaurados para garantir que o simulador
  retome sua operação \textit{exatamente do mesmo ponto} que estava antes de ser
  pausado?
\end{frame}

\begin{frame}{Pausando processo}
  \large
Suponha que um processo, num computador, possa ser
  ``pausado'' antes do computador ser desligado, e
  depois retome quando o computador é ligado novamente. Quais elementos (de
  hardware e software) devem ser restaurados para garantir que o processo
  retome sua operação \textit{exatamente do mesmo ponto} que estava antes de ser
  pausado?
\end{frame}

\begin{frame}{Mudando de processos}
  \large
  Suponha que um processo A, num computador, será ``pausado'' para que outro
  processo B passe a ser executado. Mostre todos os passos necessários para
  passar do processo A ao processo B, e depois para retomar o processo A, sem
  que haja perda de informação nesse processo.
\end{frame}



\end{document}
