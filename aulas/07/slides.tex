\documentclass{beamer}
%
% Choose how your presentation looks.
%
% For more themes, color themes and font themes, see:
% http://deic.uab.es/~iblanes/beamer_gallery/index_by_theme.html
%
\mode<presentation>
{
  \usetheme{Madrid}      % or try Darmstadt, Madrid, Warsaw, ...
  \usecolortheme{default} % or try albatross, beaver, crane, ...
  \usefonttheme{default}  % or try serif, structurebold, ...
  \setbeamertemplate{navigation symbols}{}
  \setbeamertemplate{caption}[numbered]
}

\usepackage[english]{babel}
\usepackage[utf8x]{inputenc}
\usepackage{graphicx}
\usepackage{array}

\title[07-Yacc]{EA879 -- Introdução ao Software Básico\\Yacc}
\author{Tiago F. Tavares}
\institute{FEEC -- UNICAMP}
\date{Aula 07 -- 15/agosto/2017}

\begin{document}

\begin{frame}
  \titlepage
\end{frame}

% Uncomment these lines for an automatically generated outline.
%\begin{frame}{Outline}
%  \tableofcontents
%\end{frame}

\section{Introdução}

\begin{frame}{Objetivos}
  \Large
  \begin{itemize}
    \item Entender o que é e como usar o Yacc.
    \item Entender a sintaxe do Yacc.
    \item Entender como compilar programas usando Yacc.
    \item Construir uma pequena aplicação usando Yacc.
    \item Entender o que é análise sintática.
  \end{itemize}
\end{frame}

\begin{frame}[fragile]{Exercício}
  \centering
  Resolva a expressão abaixo passo a passo:

  $5 + 3 \times (3 + 2) - 10$

  \begin{enumerate}
  \item Em que ordem a expressão foi resolvida? Por que?
  \item Quantas regras de formação diferentes foram usadas?
  \end{enumerate}

\end{frame}

\begin{frame}{Exercício}
  Ordene as instruções abaixo de forma a permitir resolver expressões.
  \begin{enumerate}
    \item Resolver somas e subtrações, na ordem que aparecem, gerando expressões equivalente.
    \item Resolver multiplicações e divisões, na odem que aparecem, gerando expressões equivalente.
    \item Resolver expressões entre parênteses, na ordem que aparecem, gerando
      expressão equivalente.
  \end{enumerate}

\end{frame}

\begin{frame}{Live Coding}
  \centering
  \Large
  Yacc é uma ferramenta que permite programar regras de formação baseadas nos
  tokens gerados pelo Lex, e associar comportamentos específicos à aplicação de
  regras de formação específicas.
\end{frame}

\begin{frame}{Exercício}
  \centering
  \large
  Proponha uma modificação ao programa feito ao vivo para incorporar parênteses
  à calculadora.
\end{frame}

\begin{frame}{Análise sintática de C}
\large
  Mostre as regras de formação (gramática livre de contexto!) capazes de reconhecer:
  \begin{enumerate}
    \item Expressões matemáticas vãlidas
    \item Operações de atribuição
    \item Variáveis que já foram declaradas anteriormente
  \end{enumerate}
\end{frame}



\end{document}
