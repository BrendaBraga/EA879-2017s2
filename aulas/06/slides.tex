\documentclass{beamer}
%
% Choose how your presentation looks.
%
% For more themes, color themes and font themes, see:
% http://deic.uab.es/~iblanes/beamer_gallery/index_by_theme.html
%
\mode<presentation>
{
  \usetheme{Madrid}      % or try Darmstadt, Madrid, Warsaw, ...
  \usecolortheme{default} % or try albatross, beaver, crane, ...
  \usefonttheme{default}  % or try serif, structurebold, ...
  \setbeamertemplate{navigation symbols}{}
  \setbeamertemplate{caption}[numbered]
}

\usepackage[english]{babel}
\usepackage[utf8x]{inputenc}
\usepackage{graphicx}
\usepackage{array}

\title[06-Gramática Livre de Contexto]{EA879 -- Introdução ao Software Básico\\Gramática Livre de Contexto}
\author{Tiago F. Tavares}
\institute{FEEC -- UNICAMP}
\date{Aula 06 -- 15/agosto/2017}

\begin{document}

\begin{frame}
  \titlepage
\end{frame}

% Uncomment these lines for an automatically generated outline.
%\begin{frame}{Outline}
%  \tableofcontents
%\end{frame}

\section{Introdução}

\begin{frame}{Objetivos}
  \Large
  \begin{itemize}
    \item Entender quais problemas não podem ser resolvidos por gramáticas
       regulares.
    \item Entender gramáticas livres de contexto como regras de formação.
    \item Entender como resolver expressões matemáticas com GLCs.
  \end{itemize}
\end{frame}

\begin{frame}{Expressões Matemáticas}
\large
  Como é possível saber que $56 + 93$ resulta em um inteiro? Ordene as etapas do
  raciocínio!
  \begin{enumerate}
    \item podemos trocar $56 + 93$ pelo valor $149$

    \item O sinal $+$ denota uma operação matemática.
    \item Na soma, podemos trocar os valores dos operandos e o sinal $+$ pelo
      valor do resultado da adição.
    \item $56$ e $93$ podem ser identificados como números inteiros.

    \item Definimos arbitrariamente a operação $+$ como a adição matemática
      clássica.
    \item $149$ pode ser identificado como inteiro
  \end{enumerate}
\end{frame}

\begin{frame}{Gerando e resolvendo somas}
\large
\centering
  \includegraphics[width=0.8\textwidth]{dot/grafo_soma.pdf}

  Trata-se de uma \textit{regra de formação}.
\end{frame}


\begin{frame}{Gerando e resolvendo somas}
\large
\centering
  \includegraphics[width=0.8\textwidth]{dot/grafo_soma2.pdf}

Para gerar expressões, aplicamos regras no sentido direto. Para resolvê-las,
  aplicamos as regras de formação no sentido reverso.
\end{frame}

\begin{frame}{Gramáticas Livres de Contexto}
\large
\centering
  \begin{itemize}
  \item Regras de formação de strings definem uma gramática!
  \item Não é uma Gramática Regular (não pode ser expressa por expressões
    regulares)
  \item São chamadas ``Livres de Contexto'' porque...
  \item <2-> ... a aplicação das regras em um nível da árvore não depende das
    regras que foram aplicadas nos outros níveis da árvore, apenas dos
      resultados.
  \end{itemize}
\end{frame}



\begin{frame}{Exercício}
\Large
\centering
Desenhe as árvores sintáticas para resolver as expressões abaixo. Evidencie a
  ordem em que as regras de formação foram aplicadas.
  \begin{enumerate}
    \item $5 + 7 + 8 + 10$
    \item $5 + 7 + 8 \times 10$
  \end{enumerate}
\end{frame}

\begin{frame}{Exercício}
\Large
\centering
Desenhe as árvores sintáticas para resolver as expressões abaixo. Como é
  possível lidar com o tipo \textit{float}?
  \begin{enumerate}
    \item $5.3 + 7 + 8 + 0.5$
    \item $5 + 7.3 + 8 \times 0.5$
  \end{enumerate}
\end{frame}

\begin{frame}{Exercício}
\Large
\centering
As expressões $5 + 6$, $5$, $3.14$ e $8 \times 0.5$ são todas válidas. Como é
  possível definir uma (ou mais de uma?) regra de formação que, aplicada,
  permita saber que todas são expressões válidas?
\end{frame}


\begin{frame}{Problema}
\Large
\centering
Crie uma regra de formação que permita gerar sequências de caracteres que são
  parênteses balanceados. Após, crie novas regras para contemplar sequências
  tais como ``()()'' ou ``()(())()()(((())))''.
\end{frame}



\end{document}
